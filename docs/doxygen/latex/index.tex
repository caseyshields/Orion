\hypertarget{index_intro_sec}{}\section{Introduction}\label{index_intro_sec}
Star hunting software. Contains a planning component for loading astrometric catalogs and selecting suitable stars. The other component is a tracker which transforms catalog coordinates into a useable format and broadcasts them to client(s) over T\+A\+TS. The \href{http://aa.usno.navy.mil/software/novas/novas_info.php}{\tt Novas 3.\+1} library is used to perform the transformations. The first part of \href{http://www-kpno.kpno.noao.edu/Info/Caches/Catalogs/FK5/fk5.html}{\tt F\+K5} or \href{http://cdsarc.u-strasbg.fr/viz-bin/Cat?I/264}{\tt F\+K6} is currently used as a catalog. The front end is still in the conceptual phases, however a \href{https://caseyshields.github.io/starlog/index.html}{\tt UI experiment} has been written in \href{https://d3js.org/}{\tt D3}.\hypertarget{index_component_sec}{}\section{Components}\label{index_component_sec}

\begin{DoxyItemize}
\item vmath \+: utility library of vector and spherical math routines
\item novas \+: U\+S\+NO\textquotesingle{}s astrometric software package
\item fk6 \+: loads raw F\+K6 catalogs into a catalog
\item catalog \+: module for filtering and sorting desired stars.
\item tracker \+: performs coordinate transforms according to the current time and location on earth
\item sensor \+: a dummy server which simulates a slaved sensor
\item orion \+: main program which integrates all features
\end{DoxyItemize}\hypertarget{index_vmath}{}\subsection{V\+Math}\label{index_vmath}
\subsection*{vmath}

\subsection*{Requirements}


\begin{DoxyItemize}
\item \mbox{\hyperlink{struct_orion}{Orion}} is developed in C\+Lion using Min\+GW 
\end{DoxyItemize}